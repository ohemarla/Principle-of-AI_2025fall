\documentclass[11pt]{article}

\usepackage[a4paper]{geometry}
\geometry{left=2.0cm,right=2.0cm,top=3cm,bottom=2.5cm}
\usepackage{setspace}
\setstretch{1.5}
\usepackage{ctex}
\usepackage{amsmath,amsfonts,graphicx,amssymb,bm,amsthm}
\usepackage{algorithm,algorithmicx}
\usepackage[noend]{algpseudocode}
\usepackage{fancyhdr}
\usepackage{booktabs}
\usepackage{array}
\usepackage{listings}
\usepackage{listingsutf8}
\usepackage{xcolor}
\lstset{
	language=Python,
	basicstyle=\ttfamily\small,
	numbers=left,
	numberstyle=\tiny\color{teal!70!black},
	showstringspaces=false,
	breaklines=true,
	frame=single,
	rulecolor=\color{black},
	framerule=0.6pt,
	framesep=4pt,
	xleftmargin=6pt,
	backgroundcolor=\color{gray!4},
	inputencoding=utf8,
	columns=fullflexible,
	keywordstyle=\bfseries\color{blue!70!black},
	commentstyle=\itshape\color{green!50!black},
	stringstyle=\color{red!60!black},
	identifierstyle=\color{violet!70!black},
	emphstyle=\bfseries\color{orange!80!black},
	tabsize=4
}
\newcolumntype{M}[1]{>{\centering\arraybackslash}m{#1}}

\begin{document}
	\pagestyle{fancy}
	\lhead{\kaishu 中国科学院大学}
	\chead{}
	\rhead{\kaishu 2025年秋季学期\qquad 人工智能原理:模型与算法}
	
	\begin{center}
		{\LARGE \bf 课程作业1} 
	\end{center}
	\begin{center}
		\large\kaishu{尧祥临 202518023406038 前沿交叉科学学院}
	\end{center}
	\section{任务介绍}
	在老师给的利用MCTS实现的黑白棋代码框架基础上,需要完成以下两个任务:
	\begin{itemize}
		\item[$\clubsuit$] 在蒙特卡洛树搜索中,补充实现UCB算法以选择最佳节点进行扩展和模拟。
		\item[$\clubsuit$] 原代码使用Roxanne策略进行模拟,尝试替换为随机模拟策略,并比较两种策略的效果差异。
	\end{itemize}
	\section{任务一:补充实现UCB算法}
	\subsection{UCB算法介绍}
	任务是要在蒙特卡洛树搜索的节点选择中,如果节点没有未被访问过,则直接选择该节点;但是如果节点有未被访问过的子节点,则需要使用UCB算法来选择最佳子节点进行扩展和模拟。UCB算法全称为上置信界算法(Upper Confidence Bound),其计算的公式如下:
	\begin{equation*}
		UCB(n) = \frac{U(n)}{N(n)} + C \sqrt{\frac{\log N(\mathrm{Parent}(n))}{N(n)}}
	\end{equation*}
	其中,$UCB(n)$表示节点$n$的$UCB$值,$U(n)$表示节点$n$获胜的次数,$N(n)$表示节点$n$被选中的次数,$\mathrm{Parent}(n)$表示节点$n$的父节点。这里加号前面代表该节点当前的平均奖励,加号后面则给出了一个置信度上界。$C$是一个平衡系数,通常取值为$\sqrt{2}\approx 1.414$,但是这里为了去进一步探索平衡系数的取值对于模型效果的影响,我们将其设置为可调节的参数。该算法需要计算出各个节点的$UCB$值,然后选择值最高的节点进行扩展和模拟。这一算法的核心思想就是在选择子节点的时候,自动权衡探索(exploration)和利用(exploitation),从而提高搜索效率和效果。
	\subsection{代码实现}
	\begin{lstlisting}[label={lst:ucb},language=Python]
for k in node.child.keys():
    if node.child[k].n == 0:    # 如果子节点未被访问过,则直接选择该节点
        best_move = k
        break
    else:
        '''作业任务1:补充实现UCB算法'''
        UCB = (node.child[k].w / node.child[k].n) + self.const_ucb * sqrt(log(node.n) / node.child[k].n)   # Exploitation + Exploration
        if UCB > best_score:    # 根据UCB算法,需要选择UCB值最大的节点
            best_score = UCB
            best_move = k	
	\end{lstlisting}
	
	这里我设置了\lstinline|self.const_ucb|为可调节的参数,来表示UCB算法中的平衡系数。
	\section{任务二:替换模拟策略}
	\subsection{Roxanne策略与随机策略}
	在原代码中,模拟搜索过程中AI使用的是Roxanne策略,这是一种启发式的策略,提前提供了一个较为合理的落子顺序。我们将其替换为随机策略,即在每一步模拟中随机选择一个合法的落子位置,为此需要另外构造一个随机模拟策略类\lstinline|RandomPlayer|,并在\lstinline|AIPlayer|类中将模拟策略从\lstinline|RoxannePlayer|替换为\lstinline|RandomPlayer|。
	\subsection{代码实现}
	\begin{lstlisting}[label={lst:random},language=Python]
class RandomPlayer(object):
    ''' 构造随机落子策略类,与Roxanne策略类相对应,结构相似 '''
    def __init__(self, color):
        """
        随机落子策略初始化
        :param color: 执棋方
        """
        self.color = color

    def random_select(self, board):
        """
        采用随机策略选择落子策略
        :return: 落子策略
        """
        action_list = list(board.get_legal_actions(self.color))
        if len(action_list) == 0:
            return None
        else:
            return random.choice(action_list)   # 从合法落子位置中随机选择一个位置,从而实现随机落子策略

    def get_move(self, board):
        """
        采用随机策略进行搜索
        :return: 落子
        """
        if self.color == 'X':
            player_name = '黑棋'
        else:
            player_name = '白棋'
        action = self.random_select(board)
        return action

class AIPlayer(object):
    ''' 蒙特卡罗树搜索智能算法 '''
    def __init__(self, color, time_limit = 2, const_ucb = 1.414):
...
        self.sim_black = RandomPlayer('X')
        self.sim_white = RandomPlayer('O')
	\end{lstlisting}

	可以看到,新构造的\lstinline|RandomPlayer|类与原有的\lstinline|RoxannePlayer|类结构相似,但是该类中落子策略函数变成\lstinline|random_select|,这里返回的是合法落子位置中随机选择的一个位置,从而实现随机落子策略。
	\section{拓展与分析}
	虽然已经补充了UCB算法并用随机策略替换了Roxanne策略,但是为了更好的理解UCB算法中的平衡系数$C$对于模型效果的影响以及不同模拟策略的效果差异,这里我对\lstinline|AIPlayer|类和最终的运行部分的代码做了部分调整和拓展。
	\begin{lstlisting}
class AIPlayer(object):
    ''' 蒙特卡罗树搜索智能算法 '''
    def __init__(self, color, time_limit = 30, const_ucb = 1.414, Player = RandomPlayer):
        """
        蒙特卡洛树搜索策略初始化
        :param color: 执棋方
        :param time_limit: 蒙特卡洛树搜索每步的搜索时间步长
        :param tick:记录开始搜索的时间
        :param sim_black, sim_white: 模拟黑白棋双方落子策略
        :param const_ucb: UCB算法中的平衡系数,默认设置为1.414
        :param player: 采用的落子策略,默认为Random策略,可以调整为随机策略
        """
        self.time_limit = time_limit
        self.tick = 0
        self.sim_black = Player('X')	# 默认采用Random策略进行模拟搜索
        self.sim_white = Player('O')
        self.color = color
        self.const_ucb = const_ucb	# 默认设置为1.414
...
# 人类玩家黑棋初始化
#black_player =  HumanPlayer("X")
# 也可以选择一个AI玩家作为黑棋
black_player = AIPlayer("X", const_ucb = 2, Player = RoxannePlayer,)
# AI 玩家 白棋初始化
white_player = AIPlayer("O", const_ucb = 1.414, Player = RoxannePlayer)
# 游戏初始化,第一个玩家是黑棋,第二个玩家是白棋
game = Game(black_player, white_player)
# 开始下棋
game.run()
	\end{lstlisting}

	这里在\lstinline|AIPlayer|类的初始化函数中增加了一个可调节的参数\lstinline|Player|,用来表示采用的模拟策略类,可以选择\lstinline|RoxannePlayer|或者\lstinline|RandomPlayer|。同时在最终的运行部分代码中,可以通过调整\lstinline|const_ucb|和平衡系数$C$的取值,来观察不同平衡系数和模拟策略对于模型效果的影响。为了进行简单的比较分析,这里分为两组:一组采用Roxanne策略,另一组采用随机策略,每组分别设置$C=1, 1.414, 2$,然后让AI进行循环对战,记录胜负情况,两两之间只进行两局对战,仅交换先后手位置。经过测试,结果如表\ref{tab:4.1}所示:

	\begin{table}[htbp]
		\centering
		\caption{黑白棋对局结果(列为白棋配置,行为黑棋配置)}
		\label{tab:4.1}
		\begin{tabular}{lcccccc}
			\toprule
			& Rox, 1 & Rox, 1.414 & Rox, 2 & 随机, 1 & 随机, 1.414 & 随机, 2 \\
			\midrule
			Rox, 1   & 白胜 & 白胜 & 黑胜 & 黑胜 & 黑胜 & 白胜 \\
			Rox, 1.414 & 黑胜 & 黑胜 & 黑胜 & 黑胜 & 白胜 & 黑胜 \\
			Rox, 2   & 黑胜 & 白胜 & 黑胜 & 黑胜 & 白胜 & 黑胜 \\
			随机, 1       & 黑胜 & 黑胜 & 白胜 & 黑胜 & 黑胜 & 黑胜 \\
			随机, 1.414     & 黑胜 & 白胜 & 白胜 & 黑胜 & 黑胜 & 白胜 \\
			随机, 2       & 黑胜 & 白胜 & 白胜 & 黑胜 & 白胜 & 黑胜 \\
			\bottomrule
		\end{tabular}
	\end{table}

	根据对局结果可以看出黑白棋的先手优势显著,黑棋整体胜率明显高于白棋。总体看来,Roxanne策略的效果优于随机策略,这和其启发式的设计有关,而随机策略在$C$较大时表现有所提升,但仍然不及Roxanne策略。而平衡参数$C$的取值在这里简单的测试中并未表现出明显的差异,Roxanne策略在黑白棋中表现更优,平衡参数$C$的影响需要进一步深入研究。
\end{document}