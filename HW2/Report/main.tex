\documentclass[11pt]{article}

\usepackage[a4paper]{geometry}
\geometry{left=2.0cm,right=2.0cm,top=3cm,bottom=2.5cm}
\usepackage{setspace}
\setstretch{1.5}
\usepackage{ctex}
\usepackage{amsmath,amsfonts,graphicx,amssymb,bm,amsthm}
\usepackage{algorithm,algorithmicx}
\usepackage[noend]{algpseudocode}
\usepackage{fancyhdr}
\usepackage{booktabs}
\usepackage{array}
\usepackage{listings}
\usepackage{listingsutf8}
\usepackage{xcolor}
\lstset{
	language=Python,
	basicstyle=\ttfamily\small,
	numbers=left,
	numberstyle=\tiny\color{teal!70!black},
	showstringspaces=false,
	breaklines=true,
	frame=single,
	rulecolor=\color{black},
	framerule=0.6pt,
	framesep=4pt,
	xleftmargin=6pt,
	backgroundcolor=\color{gray!4},
	inputencoding=utf8,
	columns=fullflexible,
	keywordstyle=\bfseries\color{blue!70!black},
	commentstyle=\itshape\color{green!50!black},
	stringstyle=\color{red!60!black},
	identifierstyle=\color{violet!70!black},
	emphstyle=\bfseries\color{orange!80!black},
	tabsize=4
}
\newcolumntype{M}[1]{>{\centering\arraybackslash}m{#1}}

\begin{document}
	\pagestyle{fancy}
	\lhead{\kaishu 中国科学院大学}
	\chead{}
	\rhead{\kaishu 2025年秋季学期\qquad 人工智能原理:模型与算法}
	
	\begin{center}
		{\LARGE \bf 课程作业2} 
	\end{center}
	\begin{center}
		\large\kaishu{尧祥临 202518023406038 前沿交叉科学学院}
	\end{center}

	\section{任务介绍}

	本次作业的目标是构建并训练一个用于图像分类任务的深度学习模型,所用数据集包含5类花卉图像,分别为daisy、dandelion、rose、sunflower和tulip,数据集已经分好成训练集和验证集以及测试集。具体任务包括模型设计与实现、训练过程分析以及结果展示。
	\section{模型架构}
	\subsection{Residual Block结构}
	\subsection{网络结构}
	\subsection{模型超参数与训练设置}
	\section{训练过程}
	\section{结果分析}

\end{document}